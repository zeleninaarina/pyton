\documentclass[12pt]{article} 

\usepackage{ucs} 
\usepackage[utf8x]{inputenc}
\usepackage[russian]{babel} 
\usepackage[left=2cm,right=2cm,top=3cm,bottom=0,5cm]{geometry}
\usepackage{ amsmath}
\usepackage{hyperref}
\date{9 марта 2018 год}
\author{А.\ И.\ Зеленина}
\title{\bf Лекции по дискретной математике } 

\begin{document}

\maketitle
\newpage
\tableofcontents
\newpage
\section{  Перестановки, размещения, сочетания}
Пусть $A= \{a_1,\ldots,a_n\}$--- конечное множество. Совокупность из $k$ элементов множества $A$ (не~обязательно различных) называется $k$--выборкой множества $ A$. Выборка называется упорядоченной, если каждомуее элементу
поставлен в~соответствие номер -- натуральное число, не~превосходящее $k$
так, что разным элементам соответствуют разные числа. Упорядоченные
выборки будем называть также наборами. Элементы упорядоченных выборок будем заключать в~круглые скобки, а элементы неупорядоченных
выборок-- в~фигурные скобки. Например, $(a_1, a_2, a_2) $и $(a_2, a_1, a_2)$ -- две различных упорядоченных выборки, а $\{a_1, a_2, a_2\}$и $\{a_2, a_1, a_2\}$-- одна и та~же неупорядоченная выборка.

Перестановкой $n$--элементного множества $A = \{a_1,\ldots , a_n\}$ называется
любой набор $(a_{i_1},\ldots, a_{i_n})$, состоящий из~элементов $A$, в~котором каждый элемент из~$A$ встречается ровно один раз. Например, у~трехэлементного
множества $\{a_1, a_2, a_3\}$ существует ровно шесть различных перестановок:
\begin{align*}
    (&a_1 ,a_2,a_3), &\qquad (&a_1 ,a_3,a_2), &\qquad (&a_2 ,a_1,a_3),\\
    (&a_2 ,a_3,a_1), &\qquad (&a_3 ,a_1,a_2), &\qquad (&a_3,a_2,a_1).
\end{align*}

Найдем число $P_n$ различных перестановок $n$--элементного множества.
Для этого из $n$~--элементного множества будем последовательно выбирать
элементы и формировать из них упорядоченную выборку: первый выбранный элемент станет первым элементом упорядоченной выборки, второй --вторым и~т.~д. Нетрудно видеть, что первый элемент можно выбрать $n$~способами. Второй элемент будет выбираться из~$(n−1)$ оставшихся элементов,поэтому его можно выбрать $(n − 1)$ способом. Продолжая выбор, заметим, что после выбора первых ~$k$ элементов останется~$(n − k)$ невыбранных элементов. Следовательно,~$(k+1)$-й элемент можно выбрать~$(n−k)$ способами.Перемножив числа способов, которыми можно выбрать первый, второй,\ldots,$(n − 1)$-й и $n$-й элементы, получим величину, равную числу способов, которыми можно упорядочить $n$-элементное множество. Таким образом,
\begin{equation}
    P_n=n\cdot(n-1)\cdot\ldots\cdot2\cdot1=n!\label{1}
\end{equation}
\mathcal{Размещением} из $n$ элементов по $k$ называется произвольная перестановка $k$-элементного подмножества $n$-элементного множества. Для обозначения числа размещений из $n$ элементов по $k$ используется символ $A^k_n$. Рассуждениями аналогичными приведенным выше при определении величины $P_n$, нетрудно показать, что
\begin{equation}
    A^k_n=n(n-1)\cdot\ldots\cdot(n-k+1)=\frac{n!}{(n-k)!}.\label{2}
\end{equation}
\mathcal{Сочетанием} из $n$ элементов по $k$ называется произвольное $k$-элементное подмножество $n$-элементного множества. Число сочетаний из $n$ элементов по $k$ обозначается через $\binom nk$ (иногда также используется символ $C^k_n$). Так~как у одного $k$-элементного подмножества существует ровно $k!$ различных перестановок, то из (\ref{2}) легко следует, что
\begin{equation}
    \dbinom nk=\frac{n!}{(n-k)!k!}=\frac
    {n(n-1)\cdot\ldots\cdot(n-k+1)}
    {k(k-1)\cdot\ldots\cdot2\cdot1}.\label{3}
\end{equation} \newpage
Из равенства (\ref{3}) легко вытекают следующие часто используемые свойства сочетаний:
\begin{equation}
    \dbinom nk=\dbinom n {n-k} , 
    &\qquad \dbinom nk=\dbinom{n-1}{k-1}+\dbinom{n-1}{k}.\label{4}
\end{equation}
Второе из этих равенств докажем также при помощи комбинаторных рассуждений. Пусть $A$-- множество всех $k$-элементных подмножеств множества $\{1, 2,\ldots, n\}$. Это множество разобъем на два класса $A_1$ и $A_2$ так, что в
первый класс отнесем все подмножества, содержащие $n$, а во второй класс-- подмножества без этого элемента.Нетрудно видеть, что $A_1$ состоит из $\binom {n-1}{k-1} $подмножеств, а $A_2$-- из $\binom {n-1}{k}$.Так как каждое $k$-элементное подмножество попадает либо в класс $A_1$, либо в класс $A_2$, то $|A|=|A_1|+|A_2|$, и, следовательно, $\binom nk =\binom {n-1}k +\binom {n-1}{k-1}$.

\mathcal{Сочетанием} с повторениями из $n$ элементов по $k$ называется неупорядоченная $k$-выборка $n$-элементного множества. Например, из~трех элементов $a_1$, $a_2$ и $a_3$ можно составить шесть сочетаний с~повторениями по~два элемента:
\begin{align*}
    a_1 a_1, \qquad a_1 a_2, \qquad a_1 a_3, \qquad a_2 a_2, \qquad a_2 a_3, \qquad  a_3 a_3.
    \end{align*}
    Каждое сочетание с повторениями из $n$ элементов по $k$ однозначно определяется тем, сколько раз каждый элемент множества входит в рассматри-
ваемое сочетание. Пусть в некоторое такое сочетание элемент $a_i$ входит $m_i$
раз, где $i = 1, 2,\ldots, n.$ Этому сочетанию поставим в соответствие набор
\begin{equation}
    \underbrace{1\ldots 1}_{m_1} 0\underbrace{1\ldots 1}_{m_2} 0\ldots\ldots 0  \underbrace{1\ldots 1}_{m_n}\label{5}
\end{equation}
из $k$ единиц, сгруппированных в $n$ блоков, и $n − 1$ нулей, разделяющих эти
блоки. В этом наборе первый блок из $m1$ единиц соответствует элементу
1, второй блок из $m_2$ единиц — элементу $a_2$, и т. д. Приведенным выше
двухэлементным сочетаниям соответствуют следующие шесть наборов:
\begin{align*}
    1100,\qquad 1010, \qquad 1001, \qquad 0110, \qquad 0101, \qquad 0011.
\end{align*}
Очевидно, что набор вида (\ref{5}) однозначно определяет соответствующее
ему сочетание с повторениями. Поэтому число $H_k$ $n$ сочетаний с повторениями из $n$ элементов по $k$ равно числун аборов из $k$ единиц и $n − 1$ нулей.
Каждый такой набор можно рассматривать как набор значений характеристической функции $k$-элементного подмножества $(n+k−1)$-элементного множества. Следовательно,
\begin{equation}
    H^k_n=\dbinom {n+k-1}k=\frac{(n+k-1)!}{(n-1)!k!}.
\end{equation}
\end{document}
